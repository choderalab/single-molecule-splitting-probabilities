\documentclass[ucb,qb3,10pt,fullfrom]{ucletter}

\usepackage{palatino}

\name{John D. Chodera}
\telephone{415.867.7384}
\email{jchodera@berkeley.edu}
%\department{Department of Chemistry}
%\fromaddress{QB3 Institute}
%260J Stanley Hall #3220\\
%University of California\\
%Berkeley, CA 94720-3220}
%\mailcode{5447}
\date{\today}

%\documentclass[10pt]{letter}
%\address{
%John D. Chodera\\
%QB3 Institute\\
%260J Stanley Hall \#3220\\
%University of California\\
%Berkeley, CA 94720-3220, USA \\
%phone: 415.867.7384\\
%email: \href{mailto:jchodera@berkeley.edu}{jchodera@berkeley.edu}\\
%\name{John D. Chodera}
%}

\usepackage{fancyhdr}
\usepackage[colorlinks=true,citecolor=blue,linkcolor=blue]{hyperref}

\def\bs{$\backslash$}


\pagestyle{fancy}
\fancyhf{}
\rfoot{\small\textsc{John D. Chodera - Cover Letter - \thepage}}

\signature{\resizebox{1.5in}{!}{\includegraphics{JDC-signature}} \\
John D.~Chodera, corresponding author
} 

\usepackage{times}
\usepackage[colorlinks=true,citecolor=blue,linkcolor=blue]{hyperref}

% We will be using PDF figures and generating PDF output.
\pdfoutput=1
\pdfcompresslevel=9
\usepackage[pdftex]{graphicx}
\DeclareGraphicsExtensions{.pdf}

%        \addtolength{\oddsidemargin}{-.875in}
%        \addtolength{\evensidemargin}{-.875in}
%        \addtolength{\textwidth}{1.75in}
%
%        \addtolength{\topmargin}{-.875in}
%        \addtolength{\textheight}{1.75in}
%
%\begin{document}
%
%\begin{letter}{To:\\MFPL Faculty Search Committee}
%
%\opening{Dear Search Committee,}


\begin{document}
\begin{letter}{Karsten Kruse, Associate Editor\\
\emph{Physical Review Letters}}

\small

\opening{Dear Editor,}

We are submitting a revised version of our manuscript:

{\bf Splitting probabilities as a test of reaction coordinate choice in single-molecule experiments}\\
by John D.~Chodera and Vijay S.~Pande.

We thank the referees for their positive feedback and constructive suggestions.
%During the review process, we also received additional feedback from colleagues, and have made a few additional edits to improve clarity and readability.
We believe that the revised manuscript is now clearer and more easily understood by a broader audience as a result.
To facilitate re-review, we have marked revised sections of the manuscript in red.
Note that figures appearing in the manuscript no longer require color.

Below, we respond point-by-point to the referee comments, clarifying how we have adapted the manuscript in response to their suggestions.
Original comments from referees appear below in color, and our response in black.

\color{blue}
{\bf Referee A}

This manuscript reports the development of a novel test for the
evaluation of the choice of reaction coordinate to be used to describe
the dynamics of the (single molecule) system in terms of Brownian
dynamics between two (or more) metastable states.

This test exploits the concept of splitting probabilities that are
calculated theoretically using the Brownian dynamics equation with the
potential of mean force (PMF) and independently estimated empirically
from experimentally determined trajectories.If the two probabilities
overlap well for a given range of the "reaction coordinate", the test
suggests that the putative reaction coordinate is a good coordinate.

The authors illustrate the application of this test to two systems.
One is a model two dimensional potential surface with two wells and
the other is a DNA hairpin jumping between two states (folded and
unfolded) with the dynamics probed by optical tweezers.

Interestingly, the authors using their test demonstrate that for the
DNA hairpin the molecular extension along the pulling direction is not
a good reaction coordinate for describing the hopping of the DNA
hairpin between these two states. Instead, a good reaction coordinate
was identified by assuming position-dependent diffusion (i.e.
position-dependent diffusion constant). The authors propose that the
presence of long DNA handles attached to the hairpin may be
responsible for disqualification of simple extension as a good
coordinate for this system.

This is a very interesting manuscript that should be of interest to
the single-molecule biophysics community.

\color{black}
We thank the referee for this enthusiasm and encouragement.

\color{blue}
I have just two minor points for the authors to consider.

a) What are the ramifications of using molecular extension for the
analysis of the dynamics of DNA hairpins, considering that extension
is NOT the appropriate reaction coordinate?

\color{black}
This depends, of course, on what properties one wishes to obtain from the data under the assumption that the observed dynamics is along a good reaction coordinate.
The consequences could be as simple as underestimating the rate constant for a two-state process or as subtle as inferring an erroneous mechanism for more complex processes.
The most obvious consequence is mistaking the location of the transition state---the point where the splitting probability $p_A = 1/2$---to be displaced from where one would expect the free energy barrier to be from examination of the histogram or potential of mean force alone.
For systems like DNA hairpins and proteins, this can have consequences for the interpretation of how ``brittle'' or ``compliant'' the folded and unfolded states are.
We have added text in this revision to better communicate this.

\color{blue}
b) In their model the authors use the same letters to denote different
variables (or constant). For example, $T$ is used for temperature but
also for time (capital $T$).

Also $\beta$ is used to denote $1/(k_B T)$ and for an auxiliary function in Eqs.~4 and 5.

\color{black}
We have made changes in this revision to avoid this collision of notation.

\color{magenta}
{\bf Referee B}
The authors develop a framework for computing the splitting
probability using experimental folding data from single molecule
pulling experiments with a constant gentle force that allows the
protein to unfold and refold many times. The authors parse a long
experimental colloid-to-colloid distance trajectory in a manner
reminiscent of the transition path theory by Vanden-Eijnden and
coworkers. From the experimental trajectory, the authors directly
estimate the coordinate dependent splitting probabilities. Then they
project the trajectory onto the pulling coordinate to obtain a free
energy landscape and a coordinate dependent diffusivity. They then
flatten the diffusivity by adaptively stretching the original
coordinate (and adjusting the free energy profile accordingly).
Finally, they use the adjoint equation to compute the splitting
probability that would result if the multidimensional dynamics were
consistent with the chosen projection. By obtaining the splitting
probability in these two ways, they test whether the pulling
coordinate is a good reaction coordinate. Their analysis confirms what
should be expected for some cases---that the pulling coordinate is not
necessarily a good reaction coordinate. To my knowledge this is the
first work to demonstrate this directly from the experimental data. 

\color{black}
We concur---to our knowledge, our manuscript is the first to demonstrate that the pulling coordinate is not necessarily a good reaction coordinate directly from experimental data.

\color{magenta}
In summary, the authors use cutting edge theoretical analysis to
demonstrate a deep mechanistic question directly from experimental
data. The work should be published after some minor corrections. 

\color{black}
We thank the referee for these words of encouragement.

\color{magenta}
My specific concerns are outlined below:

0) Equation (8) clearly has a typo in the exponent on the left hand
side. 

\color{black}
This equation has been removed while incorporating the referee's suggested changes below.

\color{magenta}
Also, I am not sure what the authors mean by necessary but not
sufficient. You are working with trajectories from the equilibrium
ensemble (on a the force-tilted landscape) and you are testing all
values of $p_B$. Is that just meant to convey that a good coordinate on
the force tilted landscape may not be a good coordinate on the native
landscape? I would agree with that, but otherwise, I don't see what
you mean. A test which sometimes accepts coordinates that are bad, but
which never fails a good coordinate, would be ``necessary but not
sufficient''. In your case---as outlined below in point (2)---I have
concerns that this test may occasionally fail coordinates that are
good. It may therefore be useful to describe these errors in the
statistics lingo of type I and type II errors.

\color{black}
We mean ``necessary but not sufficient'' in the latter sense---that the test should not erroneously indicate a good reaction coordinate is poor, but may fail to identify a bad reaction coordinate in some pathological cases.
This makes the test very useful, but it means that a failure of the test cannot rule out that the reaction coordinate may still be poor in some sense.
We address the referee's additional concerns raised in point (2) below.

\color{magenta}
1) Coordinate dependent diffusion in Langevin equations and
Smoluchowski equations date back many decades to shortly after the
original works by Langevin and Smoluchowski. For example, they appear
in very early theories of nucleation where the attachment frequency
(diffusivity along nucleus size coordinate) is assumed to be
proportional to nucleus area. The specific diffusion flattening and
Jacobian free energy transformations in this work are useful, but it
should be more clear from the references that refs 14 and 22 are
primarily novel in the choice of how coordinates should be stretched.
They are not new in showing what happens to the diffusivity and free
energy upon a non-uniform stretching/compressing transformation. By
the same token, equation (2) can be directly modified to accommodate a
coordinate dependent diffusion. The coordinate dependent diffusivity
would appear in the denominator of both integrands, as in the mean
first passage time expression. For both of these reasons, it seems
that the large emphasis on the diffusion flattening procedure is not
warranted. The authors should shorten the discussion of coordinate
dependent diffusion. That will provide more room to discuss the
physical meaning of your findings. I expect that experimentalists will
need some help interpreting the proposed test and deciding how it can
be used to learn about the mechanism. For example, could they tether
pulling handles to different locations on the protein in an effort to
find a good coordinate?

\color{black}
We had initially thought that separating the computation of position-dependent diffusion provided a more interesting story, but on further reflection, we agree with the referee, and have restructured the manuscript appropriately and added more discussion as suggested.
We agree that the result is much improved.

\color{magenta}
2) Reduction to a 1D Langevin model assumes that all degrees of
freedom other than that corresponding to the splitting probability
relax very quickly. For proteins of any significant complexity, this
is likely to be violated. Consider for example A $\rightarrow$ B by two different
"channels".
\begin{itemize}
\item A $\rightarrow$ ts1 $\rightarrow$ B.
\item A $\rightarrow$ ts2 $\rightarrow$ B.
\end{itemize}
ts1 and ts2 could have the same splitting probability with no way to
go between the two channels. Suppose the channels have the same free
energy profile and that the pulling coordinate is a good coordinate
for each of the two channels considered separately. Now suppose that
the two channels have very different mobilities near the transition
state along each channel. In this case, projection onto a 1D Langevin
model will mix the two mobilities in a way that does not describe
dynamics in either channel. Your test, I think, would indicate the
pulling coordinate is not a good reaction coordinate even though it is
good in the splitting probability sense. The authors should discuss
such problematic cases.

\color{black}
The case described above is indeed pathological, in that time-correlation functions and empirical splitting probabilities would be well-described a single 1D Langevin model with an effective diffusion constant, but some other statistics of a single long trajectory would not necessarily be described correctly by this model.
However, we would argue that the resolved coordinate is \emph{still} a good reaction coordinate in this situation, since it corresponds to \emph{the coordinate along which the rate-limiting diffusion process occurs between the two states}.
The switching between channels, which may be slow, happens \emph{within} the stable states, so cannot affect the behavior after the system leaves one stable state for the other.
We do agree that this is a difficult case, either operationally or conceptually, and have added some clarification of this issue to the \emph{Discussion}.

When we claim our test is ``necessary but not sufficient'', we are particularly concerned about situations where the \emph{average} splitting probability at a given value of the resolved coordinate matches the PMF-derived model, but the splitting probability \emph{distribution} is not well-peaked about the average.  
In such cases, the reaction coordinate would be poor (because the splitting probability distribution is not always tightly grouped), but the test would fail to identify the reaction coordinate as poor.
We have amended the \emph{Discussion} to clarify this point.

\color{magenta}
3) You should mention works by Peters, Beckham, and Trout, J. Chem.
Phys. 2007 and also Lechner, Bolhuis, and Ensing, J. Chem. Phys. 2010.
Both of these works compare directly projected splitting probabilities
to the predicted splitting probabilities from a 1D model. Their models
of the splitting probability are from likelihood maximization on
simulation data and their trajectories are not from the equilibrium
ensemble, but the nature of their test is similar to that in your
figures 2 and 4.

\color{black}
We agree with the referee that this work is highly relevant, and have added appropriate citations.

\closing{Kind regards,}
\end{letter}
\end{document}
